
To characterise the thermal potentials of boreholes, thermal response tests (TRTs) are often used when estimating the borehole resistances. Following the ASHARE guidelines, the inlet and outlet temperature at the borehole are recorded. The thermal conductivity of the borehole is the mean rate of temperature change over the natural logarithmic time of either the inlet or outlet temperature beyond the initial period of heat injection. The thermal conductivities are further used to estimate the borehole resistance and other parameters to evaluate potential borehole field designs. This method clearly does not provide enough information regarding the temperature evolution along the depth of borehole, i.e. the geothermal gradient situation of individual boreholes. Recent research have pointed out possibilities to use distributed TRTs that utilise fibre optic sensors that produces depth-specific temperature measurement along the borehole.  
The state-of-the-art methods used in estimating the thermal properties during thermal response tests (TRTs) were first proposed by Ingersol in 1954, and further expanded into ASHRAE guidelines in ASHRAE Handbooks. This method attempt to estimate the borehole resistance through three hypothetical stages of heat injection.  