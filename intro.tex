%Geothermal energy and its importance
Geothermal energy is becoming a much more popular option amongst other renewable energy sources. Its abundance in near-earth surface and reliable nature during long-term operation has attracted researchers from geophysical, geotechnical and building engineering. A particular challenging task during the design of borehole heat exchangers is the estimation of desired borehole length, which is often achieved using single in-situ thermal response tests happening at or close to the site of the project. This is often achieved by averaging the thermal conductivity of the overall borehole during the entire thermal response test, which is further used in estimating the overall borehole thermal resistance, and thus ultimately the necessary length of boreholes. The most prevalent method currently used in this process was first proposed by Ingersoll in 1954. His method helped engineers to estimate the total required length of borehole heat exchanger under some common simplifications, and three hypothetical energy injection stages and assumed ground thermal resistance over time. A few fundamental assumptions of this method included homogeneous ground (temperature and geological conditions), constant heat injection rate and therefore limited emphasis on the vertical variation of heat exchange along borehole heat exchangers. 

%!This is the niche of TRT, should consider adding on top, or simply include in the methodology part where the DTRT may or may not need to be further explained. 

%TRT1
To characterise the thermal potentials of boreholes, thermal response tests (TRTs) are often used when estimating the borehole resistances. Following the ASHARE guidelines, the inlet and outlet temperature at the borehole are recorded. The thermal conductivity of the borehole is the mean rate of temperature change over the natural logarithmic time of either the inlet or outlet temperature beyond the initial period of heat injection. The thermal conductivities are further used to estimate the borehole resistance and other parameters to evaluate potential borehole field designs. This method clearly does not provide enough information regarding the temperature evolution along the depth of borehole, i.e. the geothermal gradient situation of individual boreholes. 

%TRT2 - is usage in district systems
The state-of-the-art methods used in estimating the thermal properties during thermal response tests (TRTs) were first proposed by Ingersol in 1954, and further expanded into ASHRAE guidelines in ASHRAE Handbooks. This method attempt to estimate the borehole resistance through three hypothetical stages of heat injection. The total required length of borehole heat exchangers are therefore determined through the borehole thermal resistance determined through the in-situ TRT, which is based on the assumption of the ground being completely homogenous. For district-level systems with larger heating/cooling demands, this could mean up to 4,000 boreholes (Liu, Conference proceeding 2019). As geothermal energy sources become more popular, it might be beneficial for new geothermal installations to seek alternative solutions that requires fewer boreholes due to space constraints where fewer boreholes at larger depths should ideally yield similar heat extraction/removal results.

%Geothermal gradient
However, the homogeneity assumption completely disregards the geothermal gradient of the ground, particularly at a larger depth. This is a temperature increase along the depth of wells at the rate of an average 25 to 30 $\degree C$ per kilometer due to absorbed solar power. Under different circumstances, the geothermal gradient could vary between significantly: from as little as 1 $\degree C$ to as much as 5 $\degree C$. While the ground temperature of GSHP are often considered to be around 15 $\degree C$, the temperature at the bottom of borehole could be up to 25 $\degree C$ for a 500 m deep borehole heat exchanger(citation). However, as the existing thermal response tests only requires a single undisturbed ground temperature, this thermal potential may very much be overlooked since only the mean temperature of the ground is necessary for conventional TRT calculation. 

 Borehole heat exchangers (BHEs) used in those settings are either single U-bent or double U-bent polyethene tube heat exchangers. The configuration of two coaxial circular tubes inserted in a borehole that is either grouted or directly driven into the soil is, comparatively speaking, much less common\cite{zanchini_improving_2010}. However, the benefit of coaxial borehole heat exchangers are more obvious when working with deeper geothermal boreholes where higher temperatures may be achieved. Using low-enthalpy deep geothermal sources have often been associated with CBHEs\cite{dijkshoorn_measurements_2013}.  
 
Specifically regarding the temperature availability, some recent studies have gone beyond treating lower temperature geothermal energy as merely a steady source of heat at a lower temperature. As the geothermal boreholes can be drilled deeper and hence producing much warmer temperature output, introducing those outputs to building systems could, therefore, lead to higher coefficients of performance (COP) or even providing for district heating\cite{prandin_exergy_2010}. Such methods include either sourcing from higher temperature geothermal basins\cite{rybach_borehole_1992} or using deeper coaxial borehole heat exchangers(CBHE) to extract copious amounts of heat from the ground \cite{acuna_experimental_2008}. Increasing the number of boreholes or increasing the borehole depths are recognised as the two directions to scale up BHE installations, as identified by Rybach et al.\cite{ladislaus_rybach_shallow_1995} in the 1990s. As interests in harvesting renewable energy increased, many projects in Norway and Sweden built networks of BHEs (usually shallower than 500m) \cite{holmberg_thermal_2016}. Adding insulation for the inner pipe received much less attention due to the high investment necessary from the drilling and operational costs. Additionally, even accounting for the added thermal benefit in economic analysis, the results may still not justify the costs of increasing the depth of a borehole \cite{dijkshoorn_measurements_2013}.



%DTRT
Recent research have pointed out possibilities to use distributed TRTs that utilise fibre optic sensors that produces depth-specific temperature measurement along the borehole. This approach is currently understood as the  Distributed Thermal Response Test (DTRT). This approach acknowledges the temperature variations along the depth of borehole heat exchanger through the measurement of fiber optic cables, where the optic signals are interpreted as temperature distribution along the cables. This approach has been helping researchers improving their models and evaluation methods of borehole resistances (Meyer 2013) and analytical solutions (Walker 2015). A few pioneering researchers even proposed distributed thermal response test, which is essentially adding a distributed temperature sensing (DTS) cable to the borehole heat exchanger during the investigation. However, while most of these investigations were successful, the driving force of the differences between heat fluxes at different depths remains to be further understood: the heterogeneity of the ground is still oversimplified within the existing DTRT methods, particularly with respect to the different underground water flow and thermal properties of different ground layers. 

%This paper
This paper aims at changing this status-quo by adapting an existing analytical solution of coaxial borehole heat exchanger under the influence of geothermal gradient. We will validate the adapted model through the experimental data collected during a recent test well drilled on campus at Princeton University, which provide further validation for the model's usage in predicting the thermal responses of borehole heat exchangers under the influence of geothermal gradient. %Compare with non-gradient version?
We also aim at generating the potential changes in performance when alternative designs of coaxial borehole heat exchangers' (CBHEs) configurations and insulation levels are considered. We hope this paper is the first of many to set the ground for future comparison with DTRT methods used and designed for coaxial borehole heat exchangers proposed by McDaniel. 

The primary goal of this paper is to estimate the potential performance improvement through changing the configuration 

