Geothermal energy is gradually becoming a much more popular option amongst other renewable energy sources. Its abundance in near-earth surface and reliable nature during long-term operation has attracted researchers from geophysical, geotechnical and building engineering. A particular challenging task during the design of borehole heat exchangers is the estimation of desired borehole length, which is often achieved using single in-situ thermal response tests happening at or close to the site of the project. This is often achieved by averaging the thermal conductivity of the overall borehole during the entire thermal response test, which is further used in estimating the overall borehole thermal resistance, and thus ultimately the necessary length of boreholes. The most prevalent method currently used in this process was first proposed by Ingersoll in 1954. His method helped engineers to estimate the total required length of borehole heat exchanger under some common simplifications, and three hypothetical energy injection stages and assumed ground thermal resistance over time. A few fundamental assumptions of this method included homogeneous ground (temperature and geological conditions), constant heat injection rate and therefore limited emphasis on the vertical variation of heat exchange along borehole heat exchangers. 

Some very recent researchers have already approached these limitations by proposing an alternative method known as the Distributed Thermal Response Test (DTRT). This approach acknowledges the temperature variations along the depth of borehole heat exchanger through the measurement of fiber optic cables, where the optic signals are interpreted as temperature distribution along the cables. This approach has been helping researchers improving their models and evaluation methods of borehole resistances (Meyer 2013) and analytical solutions (Walker 2015). A few pioneering researchers even proposed distributed thermal response test, which is essentially adding a distributed temperature sensing (DTS) cable to the borehole heat exchanger during the investigation. However, while most of these investigations were successful, the driving force of the differences between heat fluxes at different depths is not very often investigated within the existing . 

Among existing solutions of temperature profile at different depths of borehole heat exchangers, the most common approach is employing the actual momentum of 

!This is the niche of TRT, should consider adding on top, or simply include in the methodology part where the DTRT may or may not need to be further explained. 
To characterise the thermal potentials of boreholes, thermal response tests (TRTs) are often used when estimating the borehole resistances. Following the ASHARE guidelines, the inlet and outlet temperature at the borehole are recorded. The thermal conductivity of the borehole is the mean rate of temperature change over the natural logarithmic time of either the inlet or outlet temperature beyond the initial period of heat injection. The thermal conductivities are further used to estimate the borehole resistance and other parameters to evaluate potential borehole field designs. This method clearly does not provide enough information regarding the temperature evolution along the depth of borehole, i.e. the geothermal gradient situation of individual boreholes. 
Recent research have pointed out possibilities to use distributed TRTs that utilise fibre optic sensors that produces depth-specific temperature measurement along the borehole.  
The state-of-the-art methods used in estimating the thermal properties during thermal response tests (TRTs) were first proposed by Ingersol in 1954, and further expanded into ASHRAE guidelines in ASHRAE Handbooks. This method attempt to estimate the borehole resistance through three hypothetical stages of heat injection.  