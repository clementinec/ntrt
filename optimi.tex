\subsection{Room for improvement}
	The proposed model, despite its relatively easier usage and customizability in contrast to existing libraries, has many constraints. A primary constraint is the lack of further customizability.
	
	It's also important to stress that this model only applies to estimation of borehole resistance measured at the in-situ TRT tests for borehole heat exchangers. This means only nearly-constant-injection-rate heat injection is considered. In addition, the flow rate and the flow direction also do not vary in our solve solution. Although it is not very common to change the flow rate or flow direction inside a CBHE, we may need to operate a CBHE more flexibly to achieve desirable temperature distribution along the DTS cable installed inside the CBHE.
	
	In order to achieve the temperature gradient as desirable as the one shown in the conceptual illustration in Figure~\ref{fg:hydro}, we need to optimize the temperature distribution and ultimately the temperature of the water coming back from the central tube. We believe it's important to also ackonwledge the importance of an analytical model that might help model the real-time temperature distribution upon heat injection/extraction for potential model predictive control algorithms. And despite lack of existing methods that allows us to do so, a combined analytical and numerical solution that stores the temperature distribution of both the water and the soil temperature from the last time step may be developed in subsequent study to evaluate these potential improvement upon the existing approach the BHEs. 
	  
\subsection{Cost implications}
	%Liu's existing paper on how to estimate the performance boost, discuss primarily the cost 
	%Drilling costs and operational costs of the CBHE
	It's also important to consider the cost implications of drilling deeper despite the potential geothermal gradient. Kavanaugh and Rafferty (2014) pointed out that the relationship between adding insulation and increasing the depth of borehole heat exchanger. In the meantime, the cost of drilling is also a factor that could impact the design of BHEs. As the cost of drilling could consist of over 50\% of the overall cost when constructing new BHEs, it is crucial to understand the implications of suggesting drilling deeper and potentially larger borehole heat exchangers.

	%Operational cost? 
	The operational cost of CBHEs vs. single or double U may be directly associated with the forms of BHE operation. With the different heat transfer capability of laiminar and turbulent flows.  Changing the CBHE configuration (particularly the annulus and cerntral pipe cross-sectional area ratio) may change the flow regime as much as changing the linear flow rates at the inteface, or pipe wall of the heat exchanger. A potential strategy would be to ensure the the flow as laminar as possible while the flow in the annulus as turbulent as possible. Maintaining the flow inside the central pipe closer to laminar while the flow inside the annulus closer to turbulent, it may be possible to limit the heat exchange between the inlet and outlet flow channel, while enhancing the heat extraction from the surrounding soil. Given a borehole deep enough to observe clear geothermal gradient with a properly insulated inner pipe, investigations like this may also be rewarding to further the usage of geothermal resources in building systems where the loads are transient and may vary significantly throughout the day. 

\subsection{Room for further investigations}
	%DTRT
	%TODO check for conflicts... sure we did not mention explicitly that we will DO TRT test somewhere? Check methodology.
	Using the existing data from our setup, an obvious question would be to compare the TRT resuts and the potential outputs from DTRT results. Using the temperature profile measured from the DTS, obtaining the depth-specific thermal conductivity through DTRT could be a very interesting direction to conduct futher analysis.

	Alternatively, as we have observed the temporal well temperature changes during the thermal response test, an interesting question to ask would be how to potentially maintain the temperature gradient inside the boreholes. With appropriate flow rate, this might be possible when the flow rate slow enough and insulation levels are high enough. With the depth-specific temperature and the inlet/outlet temperature monitored closely, it may be possible to minimizes the change of the temperature gradient along the borheoles instead of maximizing the harvested energy. This would apparently limit to the operation of coaxial borehole heat exchanger, for which there currently aren't widely-accepted hydraulics loss estimations at different flow rates available. Solving the Navier-Stokes equation for each adjacent cell has been used in some hybrid analytical-numerical solutions, but will not be ideal when operating a coaxial borehole heat exchanger. More realistically speaking, a model-predictive controller based on a simplified model, or a purely temperature-dependent black-box controller will have be used before more reasonable analytical solutions are available. 

	%Assuming the presense of such model.
	However, if a more desirable analytical model becomes available, in which the modeling of both the flow the borehole interface temperature profile can be monitored in real-time, further optimized well operations may also be possible. Strategically heating up or cooling down certain parts of the borehole, for example, may be an alternative route for future geothermal research. Operating and monitoring the coaxial borehole heat exchanger provides a unique opportunity to control for the heat flux at different depth of the borheole. With the combined results of both the DTRT and the analytical solution obtained, intentionally overheating and over-extracting certain portions of the borehole could potentially improve the seasonal performance of the heat exchanger. 
	
