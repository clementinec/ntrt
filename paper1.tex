\documentclass[3p]{elsarticle/elsarticle}


\usepackage{fixltx2e} % provides \textsubscript
\usepackage{hyperref} % provides \hypertarget
\usepackage{gensymb}
% bibliography
\usepackage{natbib}
\bibliographystyle{elsarticle/elsarticle-num}
\usepackage{custom-header}
%
\begin{document}
\begin{frontmatter}
\title{Performance of a borehole heat exchanger under the influence of geothermal gradient, an exploration of BHE optimization alternatives}
\author[Group1]{Hongshan Guo}

\ead{hongshan@princeton.edu}


\author[Group2]{Forrest Meggers\corref{cor1}}

\ead{fmeggers@princeton.edu}

\author[Group3]{James Tinjum}
\author[Group4]{Xiaobing Liu}

\address[Group1]{Andlinger Center for Energy and the Environment, Princeton University, Princeton, United States}
\address[Group2]{School of Architecture, Princeton University, Princeton, United States}


\cortext[cor1]{Corresponding author}

\begin{abstract}
This paper will primiarly focus on the possible consequenses of considering geothermal gradient when attempting to optimize the geothermal borehole heat exchangers. Based on previous studies on the distributed thermal response test, an adaptation of an existing analytical solution of coaxial borehole heat exchanger (CBHE) is proposed. We believe it is crucial to estimate the added thermal benefit from geothermal gradient, especially when there are needs to achieve better thermal performance of the boreholes. Understanding the growing demand of geothermal energy for district systems, this paper examines the possibility of incorporating the geothermal gradient in estimating the performance of borehole heat exchangers, and subsequently the optimization of borehole heat exchanger configurations through the proposed analytical model. According to our results, the adapted analytical model was adequate to estimate the thermal responses from the test well, and can be used as a tool to estimate different coaxial borehole heat exchanger configurations. We estimate adding insulation and changing diameters of CBHEs could improve the heat extraction by up to 60\%(What's the actual number?).\end{abstract}

\begin{keyword}

\end{keyword}

\end{frontmatter}

\section{Introduction}
Geothermal energy is gradually becoming a much more popular option amongst other renewable energy sources. Its abundance in near-earth surface and reliable nature during long-term operation has attracted researchers from geophysical, geotechnical and building engineering. A particular challenging task during the design of borehole heat exchangers is the estimation of desired borehole length, which is often achieved using single in-situ thermal response tests happening at or close to the site of the project. This is often achieved by averaging the thermal conductivity of the overall borehole during the entire thermal response test, which is further used in estimating the overall borehole thermal resistance, and thus ultimately the necessary length of boreholes. The most prevalent method currently used in this process was first proposed by Ingersoll in 1954. His method helped engineers to estimate the total required length of borehole heat exchanger under some common simplifications, and three hypothetical energy injection stages and assumed ground thermal resistance over time. A few fundamental assumptions of this method included homogeneous ground (temperature and geological conditions), constant heat injection rate and therefore limited emphasis on the vertical variation of heat exchange along borehole heat exchangers. 

Some very recent researchers have already approached these limitations by proposing an alternative method known as the Distributed Thermal Response Test (DTRT). This approach acknowledges the temperature variation along the depth of borehole heat exchanger by measuring the temperature 

!This is the niche of TRT, should consider adding on top, or simply include in the methodology part where the DTRT may or may not need to be further explained. 
To characterise the thermal potentials of boreholes, thermal response tests (TRTs) are often used when estimating the borehole resistances. Following the ASHARE guidelines, the inlet and outlet temperature at the borehole are recorded. The thermal conductivity of the borehole is the mean rate of temperature change over the natural logarithmic time of either the inlet or outlet temperature beyond the initial period of heat injection. The thermal conductivities are further used to estimate the borehole resistance and other parameters to evaluate potential borehole field designs. This method clearly does not provide enough information regarding the temperature evolution along the depth of borehole, i.e. the geothermal gradient situation of individual boreholes. 
Recent research have pointed out possibilities to use distributed TRTs that utilise fibre optic sensors that produces depth-specific temperature measurement along the borehole.  
The state-of-the-art methods used in estimating the thermal properties during thermal response tests (TRTs) were first proposed by Ingersol in 1954, and further expanded into ASHRAE guidelines in ASHRAE Handbooks. This method attempt to estimate the borehole resistance through three hypothetical stages of heat injection.  
\section{Method}
	\subsection{Experiment}

\subsubsection{Site and Borehole Heat Exchanger}
	%District system, estimated total demand to be satisified.
	Need for geothermal district heating/cooling demand for potentially the entire campus. It is crucial to assess the possibilities of using deeper geothermal heat exchangers. 
	
	%drilling
	The drilling of the well took place beginning August 8th, 2019. To characterize the formation layering at the drill site, a geological survey was performed immediately after the drilling. On top of the geological conditions shown in Figure~\ref{fg:hydro}, the tremmie 
	
	%Hydro-geological conditions.	
	The hydro-geological makeup of the well is 
	\begin{figure}
	\centering
	\includegraphics[height=0.5\textwidth]{data/geology_cbhe.png}
	\caption{Hydro-geological condition estimated through USGS survey immediately taken after the drilling of the well estimated at 1440 ft(438.9 m).}\label{fg:hydro}	
	\end{figure}
	
\subsubsection{Experimental Setup}
	To characterise the thermal potentials of boreholes, thermal response tests (TRTs) are often used when estimating the borehole resistances. Following the ASHARE guidelines, the inlet and outlet temperature at the borehole are recorded. The thermal conductivity of the borehole is the mean rate of temperature change over the natural logarithmic time of either the inlet or outlet temperature beyond the initial period of heat injection. The thermal conductivities are further used to estimate the borehole resistance and other parameters to evaluate potential borehole field designs. This method clearly does not provide enough information regarding the temperature evolution along the depth of borehole, i.e. the geothermal gradient situation of individual boreholes. Recent research have pointed out possibilities to use distributed TRTs that utilise fibre optic sensors that produces depth-specific temperature measurement along the borehole.  
	
	The state-of-the-art methods used in estimating the thermal properties during thermal response tests (TRTs) were first proposed by Ingersol in 1954, and further expanded into ASHRAE guidelines in ASHRAE Handbooks. This method attempt to estimate the borehole resistance through three hypothetical stages of heat injection.  
	
	To estimate the thermal conductivity of the formation, a thermal response test was performed on the drill site on September 10th, 2019. The test follows the guidelines recommended by the Americal Society of Heating, Refrigeration and Air-Conditioning Engineers (ASHRAE) in its HVAC Applications Handbook, Geothermal Energy Chapter. The borehole was uniformed grouted from the bottom to the top via premie pipe, and had a delay of more than five days between loop installation and test startup. The undisturbed formation temperature was estimated through the recorded temperature from the fiber optic cable installed along and around the borehole heat exchanger. The duration of the test was 48.5 hours, where the data (inlet, outlet temperature and heat injection rates) was collected every five minutes.  The heat injected was estimated to be 51 to 85 Btu/hr (or 15 to 25 W) per foot of the borehole, which averaged to be approximately 30.5 kW total heat input throughout the test. 

	\begin{figure}
	\centering
	\includegraphics[width=0.5\textwidth]{data/TRTraw}	
	\caption{Temperature measured at the inlet and outlet of the well site.}\label{fg:raw}
	\end{figure}
	
	For the purpose of the conventional TRT the project procured, the data collected by the company performing the test first undergone the procedures of a conventional TRT calculation. The relatively steady slope shown in Figure~\ref{fg:raw} of both the inlet and outlet temperatures since the 10th hour of the operation can be used to determine the overall thermal conductivity of the borehole. This can be achieved by taking the natural log time rate of change of the temperature of either the inlet and the outlet, and can be further expanded into the thermal diffusivity of the formation through an estimated heat capacity of the borehole. The estimated heat capacity of the bore can be estimated through the geological conditions previously confirmed through drilling (as shown in Figure~\ref{fg:hydro}. 	
	
	The overall borehole resistance $R_b$ may therefore be estimated through Equation~\ref{eq:Rb}\cite{beier_situ_2012}.
	
	\begin{equation}
		R_b = \frac{H}{Q}\{ T(t) - T_g -\frac{Q}{4\pi \lambda_g H} [E_i(\frac{r_b}{4 \alpha_g t})]   \}\label{eq:Rb}
	\end{equation}
	
	However, there are some inherent limitations of this method. First and foremost is the homogeneity assumption of the ground. The conventional method clearly assumes the ground to be at a homogeneous temperature along the depth of the heat exchanger, where the thermal conductivity, thermal diffusivity are also constants. However, as can be observed from Figure~\ref{fg:hydro} produced from the USGS measurement, it was clearly not the case. In particular, the hard rock or quartz at the bottom of the borehole will likely lead to a much larger thermal conductivity of the borehole, allowing the working fluid to extract more heat at the same flow rate. Within the scope of this paper, we will continue to use a single thermal conductivity calculated from the thermal response test (and the time log temperature gradient thereof) at the top of the heat exchanger to continue our analysis. However, for future research, it is desirable to expand this into a more detailed DTRT study.
	
	%Determined resulting thermal conductivity, rest of parameters as a table. Do we also need the thermal conductivity plot? Somewhere on the drive already I think...	
	
	\subsection{Analytical Model}
	%Is it clear that we're trying to demonstrate a model that can be experimentally validated that has geothermal gradient? Is it also clear that we're hoping to conduct CBHE configuration optimization? Is it clear that we also hope to simulate seasonal performance of the CBHE? (The last one is very, very unclear. 
	We will be adapting the analytical model from Beier et al., which was developed to predict the thermal response from a CBHE with known geological parameters. This model was developed to adopt undisturbed ground temperature measurement from fiber optic temperature sensors (also referred to as distributed temperature sensor, or DTS). 
	
	%Explain the method, and the ADAPTION~ using a different reference temperature profile, also re-wrote the code into Python to be used as a class for different configurations to have its performance estimated and validated(use a plot to show its performance comparison?).
	Typically, the overall thermal resistance of a geothermal bore can be considered as the combined resistance of the bore and the ground. The bore thermal resistance can be affected by many parameters, including the pipe material, configuration of the heat exchanger as well as the thermal conductivity of the grout/backfill in the bore annulus. The ground resistance is dependent primarily on the thermal conductivity and the diffusivity of the surrounding formation. For a vertical BHE, its thermal resistance is the combined effect of pipe resistance and bore annulus grout resistance. As Kavanaugh and Rafferty pointed out, the terms of pipe and grout resistance can be combined into a single Equation~\ref{eq:Rb}. 
	
	\begin{equation}
		R_b = R_p + R_{grt} = R_{film} + R_{tube} + R_{grt} = 
		\frac{1}{\pi d_i h_{conv}} \frac{ln(d_o/d_i)}{2\pi k_p} + \frac{ln(d_b/d_o)}{2\pi k_{grt}}\label{Rb}
	\end{equation}
	
	This equation translates the thermal resistance of the pipe to the combination of the pipe resistances and the fluid film resistance inside the pipe wall. For coaxial borehole heat exchanger, this translates to both the film resistance at the inner pipe and the borehole wall, as well as the tube thermal resistance of both the outer and inner pipes. Calculation of the tube thermal resistance is relatively straight forward as the values required includes the diameter of the tubes and the thermal conductivity of it. However, as we are more interested in varying the configuration and insulation level at the borehole as we are showing conceptually in the section of the CBHE, we need to expand the existing definition of the shunt resistance of the borehole heat exchanger. Building on the existing expressions from Kavanaugh and Rafferty (2014) as well as Beier et al. (2012), we have a new expression of the shunt resistance of the borehole $R_{12}$. 

	%shunt resistance R12, refer to the borehole section plot. highlight geothermal gradient usage
	More specifically, the shunt resistance $R_p$, or sometimes referred to as $R_{12}$ of the CBHE is a parameter that we can modify to achieve different yield from a known location with certain geological condition. 
	%Our primary goal is to achieve an easy-to-use solution that compares the heat extraction capability of different R12, Rg and Rs.
	%Config
	From the schematic diagram, it is evident that any design intervention to change the performance of a CBHE needs to happen at the level that affect the shunt resistance. This expression allows us to evaluate the thermal resistances of CBHEs, we group the contributing variables into two categories: direct and indirect. Direct variables are the diameters of the inner and outer pipes, thermal conductivity of the pipe material, and the thickness and material of the insulation material inside the inner pipe. The flow rate entering the CBHE is the indirect variable, which not only affect the convective heat transfer coefficients contributing to the film thermal resistances. To provide a more accurate description of the thermal resistance of the shunt resistance, expression of $R_{12}$ needs to be updated as Equation~\ref{eq:R12}. This includes the film thermal resistances at the heat exchanger surfaces, and the thermal resistances that represents the conductive heat transfer through the pipe and insulation materials, i.e. $R_{pw1}, R_{ins}, R_{pw2}$. More explicitply, these individual expressions can be written as Equation~\ref{eq:RR12}, where the overall shunt resistance may change with respect to different CBHE configurations. This is also expressed in Figure~\ref{fg:cbhesec}. 
	    
	    \begin{equation}
          R_{12} = R_{fi} + R_{pw1}+ R_{ins}+ R_{pw2} + R_{fo} \label{eq:R12}
        \end{equation}
        
        \begin{equation}
            Nu = \frac{hL}{k}\label{eq:Nuhx}    
        \end{equation}
        \begin{equation}
              \left\{
              \begin{aligned}
              & R_{fi} = \frac{1}{\pi d_{pi} h_{pi}}\\
              & R_{pw1} = \frac{ln(\frac{d_{pw1}}{d_{pi}})}{2\pi k_{pw}}\\
              & R_{ins} = \frac{ln(\frac{d_{pw2}}{d_{pw1}})}{2\pi k_{ins}}\\
              & R_{pw2} = \frac{ln(\frac{d_{po}}{d_{pw2}})}{2\pi k_{pw}}\\
              & R_{fo} = \frac{1}{\pi d_{po} h_{po}}
              \end{aligned}
              \right. \label{eq:RR12}
        \end{equation}
		
		\begin{figure*}[h!]
			\centering
			\includegraphics[width=0.8\textwidth]{data/CBHE_CrossSection}
			\caption{Schematic diagram of thermal resistance calculation for borehole.}\label{fg:cbhesec}
		\end{figure*}
	%These are the parameters that we are primarily interested in varying when changing the R12
	
If we may overtly simplify the entangled the relationship between the depth, performance and cost can be generalised into a statement: the further down the reach of the borehole, the higher the bottom of borehole temperature, and the larger the drilling costs and operational (pumping costs). And to improve the heat exchanger capability of any CBHE, we will therefore need to either change the configuration of the CBHE, or the location of the CBHE. This can be categorized into either shunt-resistance-related parameters and the site-specific parameters. 


%How the model works? 

	\subsubsection{Diameter, insulation level, and flow rates}
	Diameters, insulation level and flow rates are parameters that we may subjectively alter to modify the performance of borehole heat exchangers. 
	%insulation
	To better utilise the geothermal energy that is returned to the ground surface, insulating the inner pipe of CBHEs so that the heat transfer between the inner tube and annulus will not short-circuit the heat extraction in the CBHE.  
	The resulting importance of adding insulation on the inner pipe was highlighted in a few previous publications \cite{li_synthesis_2014,guillaume_analysis_2011}but was found to be not economically feasible for project proposal\cite{dijkshoorn_measurements_2013}. 
	One such study came from Dijkshoorn et al. , in which the construction and measurement of a 2500m deep well helped to validate the modelling of satisfying borehole performance over 30 years when using the borehole output to drive a climate control adsorption chiller. However, their results did not support the extra investment of adding an insulated inner pipe for the borehole - the entire project came to a halt at such.
	Additionally, despite acknowledging the possibility of added thermal benefit by inserting an insulated inner pipe, this was not further pursued due to cost constraints. In a very recent publication, the effect of changing the pipe configuration to achieve better thermal performances, but does not vary the design and environmental (specifically on the geothermal gradient) parameters component-by-component in CBHE design\cite{liu_numerical_2019}.

	
	%Depth
	\subsubsection{Geothermal gradient and thermal conductivity}
	Using the same set of method outlined by Beier et al. \cite{beier_situ_2012}, it is possible to use an arbitrary temperature profile as the the undisturbed ground temperature and solve for corresponding ground temperature. 
	
	Another modification of the method we're adapting is described as Equation~\ref{eq:bcggradient}, where the temperature distribution along the depth of CBHE can be considered constant at locations that are infinitely far away from the center of the test well.  
	
	\begin{equation}
            \frac{\partial T_{DS}}{\partial z_D}(r_D\rightarrow \infty, t_D,z_D) = g_D(z_D) \quad t_D > 0\label{eq:bcggradient}
    \end{equation}
	
	Geothermal gradient, thermal conductivity, are parameters that varies objectively, but varies significantly enough spatially that could also affect the performance of a CBHE. Addressing the presence of geothermal gradient is therefore extremely important when the proposed boreholes are deeper rather than shallower. Existing studies show that geothermal gradient may vary between 1 to 5 Kelvin per hundred meters' depth\cite{holmberg_thermal_2016,shrestha_assessment_2018}.

	%Validation of the analytical model
	
	%Comparison with the constant ground temperature solution
		
	%Thermal resistances. Look at the GTRI report also for a good reference.
\subsection{Analytical Solution}
	For each time step, a new analytical solution can be solved along different t, r and depth in CBHE. Following the same set of method 
	

	
	%Python
	As we are interested in the possible benefits of designing CBHEs better through different combinations of configurations, we are primarily interested in creating a lightweight algorithm that allows us to compare the expected thermal response outcomes (particularly the thermal resistances or heat extracted) between the different configurations. We therefore adapted the analytical method from Beier et al. \cite{beier_borehole_2013} with the following modifications: expanded the shunt resistance expression to allow for extra thermal insulation inside the inner pipe; adopted geothermal gradient into the undisturbed ground temperature during the solving of the target temperature function and translated the original analytical model from MathCAD into Python as a class that allows parallel comparison of the resulting thermal resistance of the boreholes.

	Therefore, for every time step during a simulated hypothetical time step, the corresponding analytical solution can be calculated for a given radius, depth and time since operation. Using Laplace transformation, the heat exchange within the central and annular flow can be solved using the Navier-Stokes equation with boundary and initial conditions. As all the variables were converted into dimensionless form, including the time component, the analytical solution requires an inverse Laplace transform to calculate temperatures in the time domain. The detailed solution of the energy equations using the initial and boundary conditions can be found in the appendix for the annulus as inlet scenario. We used the Stehfast algorithm following Beier's example in his 2013 paper\cite{beier2013} to perform the inverse laplace transformation. 

\subsection{Validation of adapted model}
	To confirm the validity of our adapted model, we want to compare the temperature profile that we can create by modeling the temperature distribution inside the CBHE. The parameter we used as the model input are as the followings shown in Table~\ref{tb:inputs_val}. 
	
	\begin{table}[h!]
            \begin{center}
            % \small\addtolength{\tabcolsep=0.11cm}{-5pt}
            % \scalbox{0.7}{
            \tabcolsep=0.11cm
%            {\singlespacing}
            \begin{tabular}{lcl}
                \hline
                Parameter & Symbol & Value\\
                \hline
                Borehole radius & $r_{b}$ & 115 mm\\
                Active heat exchanger length & L & 170 m\\
                Inner pipe outer radius & $r_{po}$ & 40 mm\\
%                Inner pipe wall thickness & $r_{po}$-$r_{pi}$ & 2.4 mm\\
                External pipe outer radius & $r_{eo}$ & 114 mm\\
                Inner pipe thickness & $d_{pp}$ & 2.4 mm\\
                External pipe thickness & $d_{ep}$ & 0.4 mm\\
                Inner and external pipe wall thermal conductivity & $k_{pp}$,$k_{ep}$ & 0.40 W/(K$\cdot$m)\\
                Ground thermal conductivity & $k_{s}$ & 3.15 W/(K$\cdot$m)\\
                Ground volumetric heat capacity & $c_{s}$ & 2.24 $\times$ $10^6$ J/(K$\cdot$$m^3$)\\
                Water flow rate & w & 0.58 $\times 10^{-3}m^3/s$\\
                Water density (at 15$^{\circ}$C) & $\rho$ & 999 kg/$m^3$\\
                Water volumetric heat capacity (at 15$^{\circ}$C) & $c_{w}$ & 4.19 $\times$ $10^6$ J/(K$\cdot$$m^3$)\\
                Water thermal conductivity (at 15$^{\circ}$C) & $k_{w}$ & 0.59 W/(K$\cdot$m)\\
                Water dynamic viscosity (at 15$^{\circ}$C) & $\mu_{w}$ & 1.138 $\times$ $10^{-3}$ kg/(m$\cdot$s)\\
%                Water Prandtl number & Pr & 8.09 \\
%                Heat input rate & Q & 6360 $W$\\
                Reference soil surface temperature & $T_{rs}$ & 8.9$^\circ$C\\
                % Average ground temperature & $T_{s}$ & 8.4$^\circ$C\\
                Nondimensional temperature of soil & $T_{DS}$ &\\
                Nondimensional temperature of inlet/outlet & $T_{D1},T_{D2}$ &\\
                \hline
            \end{tabular}
            \end{center}
            \caption{\label{table 2}Nomenclature, input parameters used in determining the performance of CBHE.}\label{tb:inputs_val}
	\end{table} 
	

\section{Results}
	\subsection{Measured results from TRT}
	

\section{Discussion and Further Optimization}
	\subsection{Room for improvement}
	The proposed model, despite its relatively easier usage and customizability in contrast to existing libraries, has many constraints. A primary constraint is the lack of further customizability.
	
	It's also important to stress that this model only applies to estimation of borehole resistance measured at the in-situ TRT tests for borehole heat exchangers. This means only nearly-constant-injection-rate heat injection is considered. In addition, the flow rate and the flow direction also do not vary in our solve solution. Although it is not very common to change the flow rate or flow direction inside a CBHE, we may need to operate a CBHE more flexibly to achieve desirable temperature distribution along the DTS cable installed inside the CBHE.
	
	In order to achieve the temperature gradient as desirable as the one shown in the conceptual illustration in Figure~\ref{fg:hydro}, we need to optimize the temperature distribution and ultimately the temperature of the water coming back from the central tube. We believe it's important to also ackonwledge the importance of an analytical model that might help model the real-time temperature distribution upon heat injection/extraction for potential model predictive control algorithms. And despite lack of existing methods that allows us to do so, a combined analytical and numerical solution that stores the temperature distribution of both the water and the soil temperature from the last time step may be developed in subsequent study to evaluate these potential improvement upon the existing approach the BHEs. 
	  
\subsection{Cost implications}
	%Liu's existing paper on how to estimate the performance boost, discuss primarily the cost 
	%Drilling costs and operational costs of the CBHE
	It's also important to consider the cost implications of drilling deeper despite the potential geothermal gradient. Kavanaugh and Rafferty (2014) pointed out that the relationship between adding insulation and increasing the depth of borehole heat exchanger. In the meantime, the cost of drilling is also a factor that could impact the design of BHEs. As the cost of drilling could consist of over 50\% of the overall cost when constructing new BHEs, it is crucial to understand the implications of suggesting drilling deeper and potentially larger borehole heat exchangers.

	%Operational cost? 
	The operational cost of CBHEs vs. single or double U may be directly associated with the forms of BHE operation. With the different heat transfer capability of laiminar and turbulent flows.  Changing the CBHE configuration (particularly the annulus and cerntral pipe cross-sectional area ratio) may change the flow regime as much as changing the linear flow rates at the inteface, or pipe wall of the heat exchanger. A potential strategy would be to ensure the the flow as laminar as possible while the flow in the annulus as turbulent as possible. Maintaining the flow inside the central pipe closer to laminar while the flow inside the annulus closer to turbulent, it may be possible to limit the heat exchange between the inlet and outlet flow channel, while enhancing the heat extraction from the surrounding soil. Given a borehole deep enough to observe clear geothermal gradient with a properly insulated inner pipe, investigations like this may also be rewarding to further the usage of geothermal resources in building systems where the loads are transient and may vary significantly throughout the day. 

\subsection{Room for further investigations}
	%DTRT
	%TODO check for conflicts... sure we did not mention explicitly that we will DO TRT test somewhere? Check methodology.
	Using the existing data from our setup, an obvious question would be to compare the TRT resuts and the potential outputs from DTRT results. Using the temperature profile measured from the DTS, obtaining the depth-specific thermal conductivity through DTRT could be a very interesting direction to conduct futher analysis.

	Alternatively, as we have observed the temporal well temperature changes during the thermal response test, an interesting question to ask would be how to potentially maintain the temperature gradient inside the boreholes. With appropriate flow rate, this might be possible when the flow rate slow enough and insulation levels are high enough. With the depth-specific temperature and the inlet/outlet temperature monitored closely, it may be possible to minimizes the change of the temperature gradient along the borheoles instead of maximizing the harvested energy. This would apparently limit to the operation of coaxial borehole heat exchanger, for which there currently aren't widely-accepted hydraulics loss estimations at different flow rates available. Solving the Navier-Stokes equation for each adjacent cell has been used in some hybrid analytical-numerical solutions, but will not be ideal when operating a coaxial borehole heat exchanger. More realistically speaking, a model-predictive controller based on a simplified model, or a purely temperature-dependent black-box controller will have be used before more reasonable analytical solutions are available. 

	%Assuming the presense of such model.
	However, if a more desirable analytical model becomes available, in which the modeling of both the flow the borehole interface temperature profile can be monitored in real-time, further optimized well operations may also be possible. Strategically heating up or cooling down certain parts of the borehole, for example, may be an alternative route for future geothermal research. Operating and monitoring the coaxial borehole heat exchanger provides a unique opportunity to control for the heat flux at different depth of the borheole. With the combined results of both the DTRT and the analytical solution obtained, intentionally overheating and over-extracting certain portions of the borehole could potentially improve the seasonal performance of the heat exchanger. 
	


Importance - and shortcomings of having the 

\renewcommand\refname{References}
\bibliography{refs.bib}

\end{document}
